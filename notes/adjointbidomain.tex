\documentclass[a4paper, reqno]{amsart}

\usepackage{a4wide}

% Amsmath
\usepackage{amsmath}
\usepackage{amssymb}

% Algorithms
\usepackage{algorithm}

% Figures
\usepackage{graphicx}

% Pretty tables
\usepackage{booktabs}

% Color (temp)
\usepackage{color}

% Math operators
\DeclareMathOperator{\Div}{div}
\DeclareMathOperator{\Curl}{curl}
\DeclareMathOperator{\Grad}{grad}

% Vector spaces
\newcommand{\R}{\mathbb{R}}
\newcommand{\M}{\mathbb{M}}
\newcommand{\N}{\mathbb{N}}
\newcommand{\Poly}[1]{\mathcal{P}^{#1}}

% Measures
\newcommand{\dx}{\,\mathrm{d}x}
\newcommand{\ds}{\,\mathrm{d}s}
\newcommand{\dX}{\,\mathrm{d}X}

% Project names
\newcommand{\dolfin}{\textrm{DOLFIN}}
\newcommand{\fenics}{\textrm{FEniCS}}
\newcommand{\ufl}{\textrm{UFL}}
\newcommand{\ffc}{\textrm{FFC}}

% Misc
\newcommand{\triang}{\mathcal{T}}
\newcommand{\foralls}{\forall \,}
\newcommand{\assumption}[1]{A#1}
\newcommand{\inner}[2]{\langle #1, #2 \rangle}
\newcommand{\average}[1]{[#1]}
\newcommand{\ddt}[1]{\frac{\partial #1}{\partial t}}

% Theorem styles
\newtheorem{thm}{Theorem}[section]
\newtheorem{lem}[thm]{Lemma}
\newtheorem{cor}[thm]{Corollary}
\newtheorem{example}[thm]{Example}
\newtheorem{remark}{Remark}

% Equation numbering
\numberwithin{equation}{section}

% Sizes
\newcommand{\figurewidth}{0.9\textwidth}

%
\newcommand{\heart}{\Omega^H}
\newcommand{\torso}{\Omega^T}

%------------------------------------------------------------------------------
\title{Adjoining the electrical activity in the heart and torso}

\author{Marie E. Rognes et al...}

\thanks{Center for Biomedical Computing at Simula Research Laboratory,
  P.O. Box 134, 1325 Lysaker, Norway (meg@simula.no). This work is
  supported by a Center of Excellence grant from the Research Council
  of Norway to the Center for Biomedical Computing at Simula Research
  Laboratory.}
\pagestyle{myheadings}
\thispagestyle{plain}
\markboth{MARIE E. ROGNES}{Adjoining a model for ze heart}

%------------------------------------------------------------------------------
\begin{document}
\begin{abstract}
\end{abstract}

\maketitle

\renewcommand{\thefootnote}{\arabic{footnote}}

%------------------------------------------------------------------------------

%------------------------------------------------------------------------------
\section{Introduction}

We consider the adjoint of a coupled set of transient partial
differential equations governing the electric activity in the heart
and torso.

\section{Governing equations}

This material is largely taken from~\cite{book:SundnesEtAl2006}.

We consider a body $\Omega = \heart \cup \torso$ where $\heart$ and
$\torso$ are open, disjoint domains, and such that $\partial \torso
\cap \partial \heart = \Gamma^{HT}$, and denote $\partial \torso
\backslash \partial \heart = \Gamma^T$; assuming for now that
$\Gamma^{HT}$ is connected; and in particular that $\partial \heart =
\Gamma^{HT}$.

\subsection{The bidomain equations}

For now, we consider a non-deforming heart in a domain $\heart$. The
bidomain equations of the propagation of an electrical signal in the
heart reads: find the transmembrane potential $v = v(x, t)$ and the
extracellular potential $u_e(x, t)$ such that
\begin{align}
  \label{eq:bidomain:1}
  \ddt{v} - \Div (M_i^\ast \Grad v + M_i^{\ast} \Grad u_e) &= - I_{\rm{ion}}^{\ast}
  \quad \text{in} \; \Omega_H, \\
  \Div \left (M_i^{\ast} \Grad v + (M_{i}^{\ast} + M_{e}^{\ast}) \Grad u_e \right )
  &= 0 \quad \quad \text{in} \; \Omega_H.
\end{align}
The (negative) fluxes are defined as
\begin{align}
  J_{i}(v, u_e) &\equiv M_i^\ast \Grad v + M_i^{\ast} \Grad u_e, \\
  J_{m}(v, u_e) &\equiv M_i^{\ast} \Grad v + (M_{i}^{\ast} + M_{e}^{\ast}) \Grad u_e .
\end{align}
We close the system by the boundary conditions
\begin{align}
  \label{eq:heart:bcs:essential}
  u_e &= u_T \quad \text{on} \; \partial \heart_D, \\
  \label{eq:heart:bcs:iflux}
  J_i(v, u_e) \cdot n &= 0 \quad \quad \text{on} \; \partial \heart, \\
  \label{eq:heart:bcs:mflux}
  J_m(v, u_e) \cdot n &= g_T \quad \text{on} \; \partial \heart_N,
\end{align}
where $\partial \heart_D$ and $\partial \heart_N$ denotes part of the
boundary $\partial \heart$, and $n$ is the outward pointing normal on
$\partial \heart$. If this system were to be considered alone, we
would assume these parts to be disjoint; however, as we shall couple
this system to the torso and then each of these parts will be the
whole boundary. Some initial conditions are also required for $v$.

\subsection{The ODEs}

The current $I_{\rm{ion}}^{\ast}$ in~\eqref{eq:bidomain:1} is governed
by systems of ordinary differential equations modelling the
electrophysiological behaviour of the heart cells. In general form,
these read: find the ions concentration(s) $s$ such that
\begin{equation}
  \ddt{s} = F(s, v, t) \quad \\
\end{equation}
plus some initial conditions for $s$. These ODEs will be specified
further later. And so, $I_{\rm{ion}} = I(s, v, t)$.

\subsection{The torso}

In the torso $\Omega_T$, we consider the following governing
equations. Find the something $u_T$ such that
\begin{align}
 - \Div (M_T^{\ast} \Grad u_T) = 0 \quad \text{in} \; \torso,
\end{align}
and letting $J_T \equiv M_T^{\ast} \Grad u_T$, with the boundary
conditions
\begin{align}
  \label{eq:torso:bcs:gammaT}
  J_T(u_T) \cdot n_T = 0 \quad \text{on} \; \Gamma_T, \\
  \label{eq:torso:bcs:gammaHT}
  J_T(u_T) \cdot n_T = - g_H \quad \text{on} \; \Gamma_{\rm{HT}}, \\
  u_T = u_H \quad \text{on} \; \Gamma_{\rm{HT}},
\end{align}
where $n_T$ denotes the outward pointing normal on the torso
boundary. Note that $n_T = - n$ on the heart--torso boundary
$\Gamma_{\rm{HT}}$.

\section{Weak formulation of the heart--torso system}

Integration by parts of the heart system gives: find $u_e$ and $v$
such that
\begin{align}
\inner{\ddt{v}}{\eta}
+ \inner{J_i(v, u_e)}{\Grad \eta}
- \inner{J_i(v, u_e) \cdot n}{\eta}_{\partial \heart}
&= \inner{- I^\ast_{\rm{ion}}}{\eta} \\
\inner{J_m(v, u_e)}{\Grad \xi}
- \inner{J_m(v, u_e) \cdot n}{\xi}_{\partial \heart}
&= 0
\end{align}
for all $\eta$ in some suitable space defined over $\heart$ and ditto
for $\xi$. Inserting the boundary condition~\eqref{eq:heart:bcs:iflux}
and recalling the assumption that $\partial \Omega^H = \Gamma^{HT}$,
we get
\begin{align}
\inner{\ddt{v}}{\eta}
+ \inner{J_i(v, u_e)}{\Grad \eta}
%- \inner{J_i(v, u_e) \cdot n}{\eta}_{\partial \heart}
&= \inner{- I^\ast_{\rm{ion}}}{\eta} \\
\inner{J_m(v, u_e)}{\Grad \xi}
- \inner{J_m(v, u_e) \cdot n}{\xi}_{\partial \heart}
&= 0
\end{align}
for all $\xi$ that satisfies any (homogenized) essential boundary
conditions on $u_e$ strongly.

Integration by parts of the torso system gives: find $u_T$ such that
\begin{equation}
  \inner{J_T(u_T)}{\Grad \eta} - \inner{J_T(u_T) \cdot n_T}{\xi}_{\partial \torso}
  = 0
\end{equation}
for all $\xi$ in some suitable space defined over $\torso$.
Splitting the boundary of the torso into $\Gamma^T$ (the outer surface
of the body) and $\Gamma^{HT}$ (the heart--torso boundary), we have
\begin{equation}
  \inner{J_T(u_T)}{\Grad \eta}
  - \inner{J_T(u_T) \cdot n_T}{\xi}_{\Gamma^T}
  - \inner{J_T(u_T) \cdot n_T}{\xi}_{\Gamma^{HT}}
  = 0 .
\end{equation}
Now, by~\eqref{eq:torso:bcs:gammaT}, the term over $\Gamma^T$ vanishes
and so including~\eqref{eq:torso:bcs:gammaHT} with $g_H = J_m(v, u_e)
\cdot n$ yields
\begin{equation}
  \inner{J_T(u_T)}{\Grad \xi}
  + \inner{J_m(v, u_e) \cdot n}{\xi}_{\Gamma^{HT}}
  = 0 .
\end{equation}
The two systems are then
\begin{align}
  \inner{\ddt{v}}{\eta}
  + \inner{J_i(v, u_e)}{\Grad \eta}
  &= \inner{- I^\ast_{\rm{ion}}}{\eta} \quad \foralls \eta \in V(\heart) \\
  \inner{J_m(v, u_e)}{\Grad \xi}
  - \inner{J_m(v, u_e) \cdot n}{\xi}_{\partial \heart}
  &= 0 \quad \foralls \xi \in V(\heart) \\
  \inner{J_T(u_T)}{\Grad \xi}
  + \inner{J_m(v, u_e) \cdot n}{\xi}_{\Gamma^{HT}}
  &= 0 \quad \foralls \xi \in V(\torso).
\end{align}

Now, we can combine the domains. By~\eqref{eq:heart:bcs:essential}, we
can say that $u = u_e = u_T$ (by the restrictions to each domain), and
get where $\xi$ is a test function over $\Omega$, and $\eta$ is a test
function over $\heart$. Adding the last two equations give
\begin{align}
  \inner{\ddt{v}}{\eta}
  + \inner{J_i(v, u_e)}{\Grad \eta}
  &= \inner{- I^\ast_{\rm{ion}}}{\eta} \quad \foralls \eta \in V(\heart) \\
  \inner{J_m(v, u_e)}{\Grad \xi}
  + \inner{J_T(u_T)}{\Grad \xi}
  &= 0 \quad \foralls \xi \in V(\Omega).
\end{align}

\section{Analytical test case for the bidomain equations}

To verify the correctness of the implementation of the bidomain
equations and any solvers of these, we consider a simple analytical
test case. We begin by considering the case where there are no ion
concentrations, in order words, the bidomain equations over a single
domain $\Omega$ without any additional system of ordinary differential
equations.

Let $\Omega = [0, 1]^2$, and take
\begin{align}
  v(x, t) &=  \cos(2 \pi x_0) \cos(2 \pi x_1) \sin(t) \\
  u(x, t) &= - 0.5 v(x, t)
\end{align}
Let $M_i = M_e = 1.0$. Then, with these choices:
\begin{equation}
  \Div J_m = \Div (M_i \Grad v + (M_i + M_e) \Grad u) = 0
\end{equation}
Further, $\int_{\Omega} u = 0$, and also with regard to the boundary
conditions, we have for all $t$:
\begin{equation}
  J_m(x, t) \cdot n = 0 = J_i(x, t) \quad x \in \partial \Omega
\end{equation}
As initial condition, we take $v(x, 0) = 0$, which is compatible with
the exact solutions. The suitable right-hand side in the parabolic
equation follows by direct derivation:
\begin{equation}
  -I_{\textrm{ion}} = f = \frac{\partial v}{\partial t} - \Div J_i(v, u) =
  4 \pi^2 \sin(t) \cos(2 \pi x_0) \cos(2 \pi x_1)
  + \cos(t) \cos(2 \pi x_0) \cos(2 \pi x_1)
\end{equation}

\bibliographystyle{siam} \bibliography{bibliography}

\end{document}
